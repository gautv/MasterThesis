\documentclass[12pt,twoside,a4paper]{report}

\usepackage[utf8]{inputenc}
\usepackage[T1]{fontenc}


\usepackage{amsmath,amsfonts,amssymb}
\usepackage[lined,boxed,commentsnumbered]{algorithm2e}

\title{Title to find ...}
\author{Gautier VAILLANT}

\begin{document}

\maketitle

\chapter*{Acknoledgements}

I would first like to gratefully thank Helmut Grubmüller and Bert De Groot for welcoming me in their lab. I also thank Carsten Kutzner who supervised my Internship. I also would like to Thank Bartosz Kohnke and Thomas Ullmann for providing me advice on my project.\\ 

Finally I would also like to thank Ivo Kabadschow and Andreas Beckmann who welcomed me in the Jülich Forschungszentrum to teach me the details of the FMM method. 

\chapter*{Abstract}

Simulating large pairwise interactions is a very important issue for Scientific research. It plays an important role in Astrophysics to know the dynamics of galaxies, in plasma physics or in our case in biophysics. This kind of simulations is typically with a complexity of $\mathcal{O}(N^2)$ which scales badly with the size of the system.

Some other techniques, such as the PME (Particle Mesh Ewald) and the FMM (Fast Multipole Method) are able to obtain a complexity of respectively $\mathcal{O}(N\log(N))$ qnd $\mathcal{O}(N)$.

\tableofcontents

\chapter*{Presentation of the Lab}

\chapter{Introduction to methods for computing electrostatic forces}


\section{$\mathcal{O}(N^2)$ method }

\subsection{Naive Method}

The coulombian interaction between two charged particles can be written the following way:

\begin{equation}
	\overrightarrow{F}_{A \rightarrow B} = \frac{q_A q_B \hat{r}_{AB} }{4\pi\epsilon_0|R_{AB}|^2}
	\label{coulombComplete}
\end{equation}

where $q_A $ and $q_B$ are respectively the charges of A and B, and $R_{AB}$ is the distance between $A$ and $B$.

In the thesis we will simplify the units of ~\eqref{coulombComplete} for computational reasons by just writing :

\begin{equation}
	\overrightarrow{F}_{A \rightarrow B} = \frac{q_A q_B \hat{r}_{AB} }{|R_{AB}|^2}
	\label{coulombSimplified}
\end{equation}

The first, naive way to compute electrostatic forces is the following : in order to compute the force acting on one particle, it is needed to obtain the coulombic interaction for each pair of particles.

So if we consider a set of $N$ charged particles, $N-1$ interactions are needed to compute the force acting on one specific particle. So in order to know the forces of the set of particles $N\cdot(N-1)$ operations are needed, hence an algorithmic complexity of $\mathcal{O}(N^2)$.

This gives the following algorithm:

\IncMargin{1em}
\begin{algorithm}[H]

\SetKwData{Left}{left}\SetKwData{This}{this}\SetKwData{Up}{up}
\SetKwData{Force}{force}

\SetKwFunction{ComputeForce}{computeForce}
\SetKwFunction{Union}{Union}\SetKwFunction{FindCompress}{FindCompress}
\SetKwInOut{Input}{input}\SetKwInOut{Output}{output}

\Input{A set of $N$ charged Particles}
\Output{A List of the forces for each particle}
\BlankLine

\emph{For each particle i}\;
\For{$i\leftarrow 1$ \KwTo $N-1$}{
\emph{add interaction between particle $i$ and particle $j$ }\;
\For{$j\leftarrow i+1$ \KwTo $N$}{

	\Force$[i]$  $\leftarrow$ \Force$[i]$ + \ComputeForce{$i,j$}   \;

}
}
\caption{Naive method}\label{algo_disjdecomp}
\end{algorithm}\DecMargin{1em}


The complexity of such a computation limits its use to rather small systems and is not really usable for bigger systems.

\subsection{Possible improvements}





\section{Fast Fourier Transform-Based methods}
\subsection{Ewald Summation}
\subsection{PME}

\section{Fast Summation methods}
	\subsection{Mathematical preliminaries}
	\subsection{Operators}
		\subsubsection{M2M}
		a
		\subsubsection{L2M}
		b
		\subsubsection{L2L}
		c


\chapter{Comparing FMM and PME accuracy}
\section{Presentation of GROMACS}
	\subsection{Structure of a File}
	\subsection{PME parameters}

\nocite{*}
\bibliographystyle{plain}
\bibliography{biblio} 



\end{document}

